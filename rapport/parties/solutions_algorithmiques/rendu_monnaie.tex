
\subsection{L'achat d'un architecte}
\label{canbuy}

\subsubsection{Algorithme glouton}

A rédiger

\subsubsection{Algorithme récursif}

Contrairement à un algorithme glouton, l'algorithme récursif va permettre de récupérer la meilleur façon de payer un architecte. 

Tout comme l'algorithme glouton, on commence à retirer au prix ce qui est produit par les architectes. On teste ensuite l'ensemble des combinaisons de jetons (stocké dans un marché) pour payer le prix.

A chaque fois que l'on peut utiliser une combinaison pour payer le prix, on compare cette combinaison avec la précédente pour sélectionner la meilleure (cf \ref{market_cmp}). On teste l'ensemble des des combinaisons de nb-jetons avec nb variant entre 1 et le nombre de couleurs dans le prix à payer.

L'algorithme pour tester les combinaisons de nb-jetons ressemble à cela :

\begin{lstlisting}[language=pascal, frame=single, caption={Algorithme de test des combinaison de nb-jetons}]
procedure test_combinaison(n, nb_jetons_voulus, dernier_indice: integer; var global, meilleur, test: Marche; prix: Set);
var
    nb_jetons: integer;
    i: integer;
    jeton: Jeton;
begin
    nb_jetons := taille_marche(test);
    
    if nb_jetons = nb_jetons_voulus then
    begin
        if test_est_utilisable_pour_payer(test, prix) then
            meilleur := meilleur_marche(meilleur, test);
    end
    else
    begin
        for i := dernier_indice to n do
        begin
            jeton := global[i];
            ajouter_jeton(test, jeton);
            test_combinaison(n, nb_jetons_voulus, i + 1, global, meilleur, test, prix);
            retirer_jeton(test, jeton);
        end;
    end;
end;

\end{lstlisting}

\subsubsection*{Complexité et terminaison}
\subsubsection*{Complexité}
L'algorithme de rendu de monnaie teste l'ensemble des combinaisons de jetons. 

Or :

\begin{equation}
    \sum_{k=1}^{n} \binom{n}{k} = 2^n
\end{equation}

avec n le nombre de jetons maximum pour un joueur. Tester si un marché est utilisable se fait en complexité linéraire.

On a donc une complexité exponentielle en $\theta(n2^n)$. \\

Cependant le prix maximum ne possède jamais autant de couleur que le nombre de jeton que possède un joueur. Dans notre cas de figure, le prix se limite à 3 couleurs différentes.

Ainsi on ne va tester seulement $\sum_{k=1}^{3} \binom{n}{k}$ combinaisons de jetons. La complexité de l'algorithme est donc grandement diminué et on plutôt un algorithme de complexité polynomial.

\subsubsection*{Terminaison}
