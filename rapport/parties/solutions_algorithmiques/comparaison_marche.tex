\subsection{Comparaison de deux marchés}

\label{market_cmp}

Afin de comparer deux marchés, nous utilisons une fonction \code{eff(marché)} qui associe à un marche le flottant suivant : 

$$\sum_{jeton \in marche} \frac {n\_ressources(a\_acheter \cap jeton)} {n\_ressources(a\_acheter)}$$

Ainsi, on a $eff(marche) < 1$ si on essaye de payer avec des jetons qui ne servent pas à l'achat, et on a $eff(marche) > 1$ si on utilise plus de jetons servant à l'achat qu'il ne le faut.

Afin de comparer deux marchés, on considère donc la relation d'ordre suivante : 

$$marche1 < marche2 \Leftrightarrow |1 - eff(marche1)| < |1 - eff(marche2)|$$

Cette relation d'ordre nous permet de discriminer les marchés ayant une efficacité proche de 1.

\begin{summary}
Si deux marchés ont la même efficacité, on considère que celui qui est inférieur est celui qui utilise le moins de jetons.
\end{summary}
