\section{Tests}
\label{tests}

\subsection*{Problématique}

Lorsque que le code est important et est amené a évoluer avec de nouvelles fonctionnalités, il peut devenir intéressant de tester le comportement des fonctions pour vérifier qu'elles respectent des règles de base. Dans notre cas de figure nous testons une grande majorité des fonctions de base, les autres fonctions découlant de ces dernières.

\subsection{Mise en place des tests}

Tout comme le reste du code, les tests sont séparés en sous dossiers par thème. Chaque thème possède une fonction qui regroupe les tests des fonctions associées à ce thème, qui sont alors exécutés par l'exécutable de test.


\begin{lstlisting}[frame=single, caption={Exécution des tests}]
int main(int argc, char *argv[])
{
	test_token();
	test_builders();
	test_market();
	test_guild();
	test_players();
	test_utils();
	test_skills();
 
	return EXIT_SUCCESS;
}
\end{lstlisting}
\subsection{Vérifications effectuées}

Chaque tests vérifie le comportement attendu de la fonction, et si elle respecte les règles définies en amont. 

Prenons l'exemple du test de rendu de monnaie, on se place dans le cas le plus défavorable pour notre algorithme : un prix avec plusieurs couleurs, avec de nombreuses ressources pour payer. On teste la fonction sur ce cas particulier.

On effectue les étapes suivantes :
\begin{itemize}
    \item On essaye de payer le prix avec des jetons simples 
    \item On retire un jeton au hasard et on vérifie qu'on ne peut plus payer
    \item On réessaye de payer mais cette fois ci avec des jetons complexes
    \item On vérifie que le rendu de monnaie est le meilleur
    \begin{itemize}[label=*]
        \item Pour cela on ajoute au marché de test un jeton complexe avec exactement le prix et on vérifie qu'il s'agit bien du seul jeton retourné par l'algorithme
    \end{itemize}
\end{itemize}
