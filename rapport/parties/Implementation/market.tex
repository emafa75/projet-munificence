

\subsubsection{Les marchés}
\label{market}


Les marchés servent avant tout à stocker les jetons, un marché global permettant aux joueurs de piocher des jetons est créé, puis chaque joueur peut utiliser son propre marché pour ensuite stocker ses jetons.


Nous définissons le type \code{struct market\_t} de la manière suivante : 


\begin{lstlisting}[frame=single, caption={implémentation du type struct market\_t}]
struct market_t {
	struct token_t* tokens[NUM_TOKENS];
	struct permutation_t permutation;
};
\end{lstlisting}


Nous utilisons le tableau \code{tokens} pour stocker les pointeurs de jetons présent initialement dans le marché global. \code{permutation} permet d'appliquer une permutation sur le replacement des jetons dans le marché (utilisé pour le marché global).


\subsubsection*{Particularité du marché global}


Les joueurs doivent pouvoir piocher uniquement des jetons qui sont connexes. Pour cela la fonction \code{market\_get\_linked\_tokens} permet de renvoyer l'indice du premier jeton d'un groupe de nb-jetons connexes du marché global (cf \ref{linkedtokens}). 